A collection of example scripts are available in the directory {\tt \$RAD\_DIR/examples}. These cover a variety of usage cases including:

\begin{description}
\item[{\tt rc3}]
: The files in this directory contain the inputs for running the
radiance code on monochromatic data (executable run\_mono) to repeat the
3rd test of the international radiation commission (see Benassi et al.).
These replicate a standard test of monochromatic solar
radiances in specific directions.

\item[{\tt aer\_cmp}]
: An example of using the code to run ICRCCM case 27 (LW) on AER profiles.
The files laer27... contain atmospheric profiles for ICRCCM case27
as used by the AER LBL code given in CDL format. The calculation of
fluxes can be performed using the script run27, which gives an example of
how the code is run. The files aer\_... are the results from the AER LbL code
itself.

\item[{\tt prsc}]
: An example of using prescribed optical
properties for a water cloud. The file {\tt input\_prsc} defines a simple
atmosphere and the script {\tt p1.scr} will run appropriate programs.

\item[{\tt netcdf}]
: Contains data to test the netCDF code ({\tt l\_run\_cdf}). The directory
{\tt netcdf/7460\_28} contains example netCDF input and output 
files for a 1x20 column section of Cloud Resolving Model data. The
directory {\tt netcdf/CIRC\_case6} contains input files for a case study
from the CIRC (Continual Intercomparison of Radiation Codes) project. 
{\tt README} files give instructions for testing.

\item[{\tt aerosols}]
: Contains a script to generate monochromatic single scattering properties
of the aerosols used operationally with HadGEM3.

\item[{\tt droplets}]
: Contains a script to generate cloud droplet optical properties using the
method employed for the operational spectral files.

\item[{\tt corr\_k}]
: Creates a skeleton 300 band LW spectral file, calculates example
water vapour k-terms for bands 38-40 using a cut-down HITRAN line
list, and adds foreign and self broadened continuum coefficients.
CFC-113 coefficients are also added as an example of the use of
cross-section data.

\item[{\tt prep\_data}]
: Test script for scatter\_90, scatter\_average\_90 and prep\_spec provided by
Marc Stringer and Jolene Cook at Reading.

\item[{\tt raw\_input}]
: Demonstrates the use of the raw\_input program to convert column data into
input files (CDL) for the ES code, including the conversion from units of
parts-per-million by volume to kg/kg.

\item[{\tt sp\_lw\_jm}]
: Scripts to create the entire 300 and 9 band LW spectral files from scratch
as intended for the next version of the Met Office global model (GA7).
These take a long time to run (days) and use a lot of space ($\sim$20G).

\item[{\tt sp\_sw\_jm}]
: Scripts to create the entire 260 and 6 band SW spectral files from scratch
as intended for the next version of the Met Office global model (GA7).
These take a long time to run (days) and use a lot of space ($\sim$20G).

\item[{\tt spectral\_var}]
: Adds a look-up table of solar spectral variability data to the GA7
shortwave spectral file. See section~\ref{sec_spectral_var} for further details.

\end{description}
