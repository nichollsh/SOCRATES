\section{The Structure of Spectral Files}

The file consists of a number of blocks of data, each referring to a different
physical process. The flag {\tt l\_present(i)} is set to {\tt .TRUE.} if
a block of type {\tt i} is present: not all possible blocks are required
for all calculations.

\begin{description}

\item[Block 0]
contains the number and physical natures of gases and aerosols. There are
a vast number of gases and aerosols in the atmosphere, not all of which
are relevant in all applciations. In each spectral file a subset of all
the gases is selected and indexed $1,\ldots,n$. This number is referred
to as the indexing number and is used internally by the radiation code.
There is still a need to know the physical nature of each species, and this
is recorded by the $type$ number. The array {\tt TYPE\_ABSORB} holds the
type numbers for gaseous absorbers. The meaning of these numbers is set
in the module {\tt gas\_list\_pcf}. Aerosols are indexed in a similar
way, the type numbers for these being recorded in {\tt rad\_pcf}.

\item[Block 1]
contains the limits of the spectral bands used as wavelengths in metres.
{\it Note: UM standards require the use of SI units.} In a number of UM
shortwave files, it will be observed that some bands have the same limits:
this indicates that they are not true spectral bands, and that one should
not consider the fluxes in individual bands alone, but only the sum of
the fluxes in the bands which does represent the true flux across the
specified region. For example, if band 1 is specified as running from
0.2--0.32 $\mu$m, but bands 2 and 3 both have limits 0.32--0.69$\mu$m,
it is meaningful to consider the flux in band 1 as representing the true
flux between 0.2 and 0.32 $\mu$m, but the flux in band 2 or band 3 should not
be considered alone: all that can be said is that the sum of the fluxes 
in bands 2 and 3 can meaningfully be taken as that in the region
0.32--0.69$\mu$m.

\item[Block 2]
is required only in shortwave files and contains the fraction of the solar
spectrum in each band.

\item[Block 3]
is required only in shortwave files and contains the Rayleigh scattering 
coefficients.

\item[Block 4]
contains the list of gaseous absorbers active in each band, listed by their
indexing numbers. Note that the first gas listed must be the primary
absorber in the band {\em i.e.} that which makes the greatest contribution
to the atmospheric absorption when considered alone.

\item[Block 5]
contains the $k$-fits to the gaseous transmissions.

\item[Block 6]
is required only for infra-red calculations and contains the 
coefficients of a polynomial fit to the Planck function in
each band.

\item[Block 7]
is obsolete and not present in any file used in the UM.

\item[Block 8]
contains the list of continuum absorbers in each band. In principle, there
are several species of continuum absorber, but in practice the main
continua are the self and foreign-broadened continua of water vapour.

\item[Block 9]
contains the continuum absorption coefficients in each band.

\item[Block 10]
contains parametrizations for the single scattering properties of
droplets. The file may contain data for a number of different
{\em types} of droplet. The term type is deliberately vague to
allow for flexibility: a different type may indicate a parametrization
appropriate to a different collection of droplets, say droplets in
convective clouds and stratiform clouds, a different parametrization
of the same data or different spectral averaging. Type numbers are
supplied at runtime and must be selected for the appropriate spectral
file. The details are given below. 
Parametrizations are generated 
over a range of particle sizes, so the minimum and  maximum dimensions
for which the parametrization is valid are recorded as well.

\item[Block 11]
contains data on aerosols. The selection of aerosols included is very varied
and is described for each file listed below.

\item[Block 12]
contains parametrizations for the single scattering properties of
ice crystals. The file may contain data for a number of different
{\em types} of droplet. As for water droplets, 
the term type is deliberately vague to
allow for flexibility: a different type may indicate a parametrization
appropriate to a different collection of ice crystals, say, crystals in
convective clouds and stratiform clouds, a different parametrization
of the same data, a different crystal shape, or different spectral 
averaging. Type numbers are
supplied at runtime and must be selected for the appropriate spectral
file. The details are given below.
Parametrizations are generated 
over a range of particle sizes, so the minimum and  maximum dimensions
for which the parametrization is valid are recorded as well.

\item[Block 13]
is only relevant in the longwave region and is obsolescent. It contains 
heuristic adjustments for Doppler broadening. Eventually, these will be
moved to block 5.

\item[Block 14]
specifies exclusions. In the original version of the radiation code a 
band had to be a contiguous range of frequencies, but for use in the
UM it was desirable to allow for split bands. 
This concept is most easily
explained by an example. If we specify that band 5 extends from 8 to 12
$\mu$m and band 6 from 10--11$\mu$m, and exclude band 6 from band 5, this
means that we take band 5 effectively to consist of the regions 
8--10 $\mu$m and 11--12$\mu$m. In this case, the limits for band 6 will
naturally be set as 10 and 11 $\mu$m, but band 5 will have limits of
8 and 12 $\mu$m. Exclusions are of importance in the generation of the
spectral file, but are not of such relevance in runs in the UM. If diagnostics
covering only a portion of the spectrum were defined, it would be 
necessary to know about any exclusions in order to weight the contributions
from individual bands appropriately. Split bands are used only for reasons
of efficiency.

\end{description}
