Because of their irregular shapes, calculating the scattering properties
of ice crystals is considerably more complicated than calculating the
scattering propeties of a water droplet. Whilst it is possible to
generate data using a Mie scattering code treating ice crystals as
spheres, this approach is largely out of date. The preferred approach
now is to use a database of scattering properties prepared using
appropriate off-line algorithms. 
Databases are too varied to be considered as an integral part of the 
radiation code and some coding to use a particular new database should
be expected.
