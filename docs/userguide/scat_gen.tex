The scattering properties of particles can be calculated directly
using electromagnetic theory; but even for a shape so simple as
a sphere this would be prohibitively expensive in a radiation
code. Therefore, scattering propeties are calculated outside the
radiation code and represented in a parametrized form within it.

For aerosols, water droplets and ice crystals, the procedure is
generically similar, although there are subtle nuances. Water
droplets are spherical and aerosols are at the present time 
treated as spherical. Over recent years it has become possible
to treat non-spherical ice crystals and this makes the treatment
of ice more dissimilar from the other two types of particle.

In any region of the atmosphere there will be a mixture of particles
of different sizes, so a size ditribution must be considered. For
water droplets and aerosols, simple analytic distributions such
as the log-normal or modified gamma distributions are often assumed.
Size distributions for ice crystals are much less regular and often
bimodal; we are tending now to use observational size distributions
to generate scattering data for ice crystals.
 
Having assumed a size distribution, it is necessary to calculate the
individual scattering properties of particles of a given size before
avaeraging across the size distribution. In the case of aerosols and
droplets, Mie scattering calculations are done, but for non-spherical
ice crystals, it would be too complicated to do these calculations
all at once and an external database is used. To do Mie calculations,
a file of refractive indices is required.

The generic procedure, then, is that single scattering propeties are
generated at a number of frequencies, averaged over a size distribution
using the program {\tt scatter}. These must then be averaged over the
bands of the spectral file, weighting with a solar spectrum or a
Planckian function (and including an instrumental response if necessary).
In the case of droplets and ice crystals we need to consider a range
of size distributions and generate a fit in terms of particle size.
This is done using the program {\tt scatter\_average}.

We shall now consider aerosols, droplets and ice crystals in turn. The
detailed description of the programs should also be consulted.
