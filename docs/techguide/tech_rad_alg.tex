
The monochromatic equation of transfer is used in the form
\begin{equation}
\begin{split}
(\bn . \nabla) I(\bx, \bn) & = -(k^{(s)} + k^{(a)}) I (\bx, \bn) \\
& + { k^{(s)} \over
{4 \pi} } \int_{\Omega} I(\bx,\bnp) P(\bnp, \bn) \, d\omega_{\bnp} 
+ j(\bx,\bn)
\end{split}
\end{equation}

The phase function can be rescaled using the standard prescription
\begin{align}
k^{(s)} & \rightarrow (1-f) k^{(s)}, \\
P(\bnp, \bn) & \rightarrow { {P(\bnp, \bn)-4\pi f \delta(\bnp - \bn)}
\over {1-f}} \\
\noalign{\text{\it i.~e.}}
P(\bnp, \bn)-4\pi \delta(\bnp - \bn) & \rightarrow 
{ {P(\bnp, \bn)-4\pi \delta(\bnp - \bn)}
\over {1-f}}.
\end{align}
Since this does not alter the functional form of the equation no further 
reference to rescaling wil be made here.

The phase function may be expanded in Legendre polynomials:
\begin{equation}
P(\bnp, \bn)= \sum_{l=0}^\infty (2l+1) g_l P_l (\bnp.\bn)
\end{equation}

We make use of the standard results
\begin{align}
P_l (\bnp.\bn) &= { 4\pi \over {2l+1} } \sum_{m=-l}^l Y_l^m (\bn) Y_l^{m*}
(\bnp) \\
\delta(\bnp - \bn) &= \sum_{l=0}^\infty \sum_{m=-l}^l Y_l^m (\bn) Y_l^{m*}
(\bnp)
\end{align}

It is useful to keep the direct solar beam separate, so we write:
\begin{equation}
I(\bx,\bn) = \sum_{l=0}^\infty \sum_{m=-l}^l I_{lm}(\bx) Y_l^m (\bn)
+ I_\odot \delta(\bnp - \bn_\odot)
\end{equation}

It now follows that
\begin{equation}
\begin{split}
\sum_{lm} Y_l^m (\bn) (\bn . \nabla) I_{lm}(\bx) &+ (\bn . \nabla) 
I_\odot(\bx) \delta(\bnp - \bn_\odot ) \\
&= - (k^{(s)} + k^{(a)} ) \biggl ( I_\odot(\bx) \delta(\bnp 
- \bn_\odot ) \\
&+ \sum_{lm} I_{lm}(\bx) Y_l^m (\bn) \biggr )  + \sum_{lm} j_{lm}(\bx) 
Y_l^m (\bn)\\
&+ k^{(s)} \int_\Omega  
\begin{aligned}[t]
& \left \{ \sum_{lm} I_{lm} (\bx) 
Y_l^m (\bnp) + I_\odot(\bx) \delta(\bnp - \bn_\odot ) 
\right \} \\
&\left \{ \sum_{l'm'} g_{l'} Y_{l'}^{m'*} (\bnp) Y_{l'}^{m'} (\bn) \right \} \,
d\omega_{\bnp}
\end{aligned}
\end{split}
\end{equation}

We separate the singular terms involving exposed $\delta$-functions to get
\begin{equation}
( \bn . \nabla) I_\odot (\bx) = -(k^{(s)} + k^{(a)} ) I_\odot (\bx).
\end{equation}
which may be integrated directly.

Making the assumption that the atmosphere is plane-parallel,
\begin{equation}
(\bn.\nabla) I (\bx) = n_0 \, dI_{lm}/dz,
\end{equation}
where $n_0(=n_z)$ is the zeroth component of $\bn$ in the spherical
basis (the others are
$n_\pm=\mp (n_x \pm i n_y)/\sqrt{2}$, so that
$\bn = \sum_{j=-1}^1 n_j \boldsymbol{\epsilon}_j^*$ where $\boldsymbol{
\epsilon}_{\pm 1}
= \mp ( \mathbf{e}_x \pm i \mathbf{e}_y )/\sqrt{2}$). Hence, using the
orthogonality of the $Y_l^m$,
\begin{equation}
\begin{split}
\sum_{lm} n_0 Y_l^m (\bn) { {dI_{lm} (z)} \over {dz} } &= -(k^{(s)} + k^{(a)} )
\sum_{lm} I_{lm}(z) Y_l^m (\bn) +\sum_{lm} j_{lm} Y_l^m (\bn) \\
&+ k^{(s)} \biggl \{ \sum_{lm} I_{lm}(z) g_l Y_l^m (\bn) \\
&+ I_\odot (z) \sum_{lm}
g_l Y_l^{m*} (\bn_\odot) Y_l^m (\bn) \biggr \}
\end{split}
\end{equation}

The left-hand side of this equation can be expressed as a pure function of
spherical harmonics using the recurrence
\begin{equation}
n_0 Y_l^m(\bn) =c_{lm}^+ Y_{l+1}^m (\bn) + c_{lm}^- Y_{l-1}^m (\bn)
\end{equation}
where
\begin{align}
c_{lm}^+ &= \sqrt{ { {(l+1-m)(l+1+m)} \over {(2l+1)(2l+3)} } } \\
\noalign{and} \\
c_{lm}^- &= \sqrt{ { {(l-m)(l+m)} \over {(2l-1)(2l+1)} } }
\end{align}
are the Clebsch-Gordan coefficients, $\langle l+1,m | 1,0,l,m\rangle$ and
$\langle l-1,m | 1,0,l,m\rangle$

By forming the inner product of this equation with $Y_l^m$ the individual
spherical harmonics may be separated. At the same time we introduce the
optical depth, $\tau$, and the albedo of single-scattering, $\omega$:
\begin{align}
d\tau &= -(k^{(s)} + k^{(a)}) \, dz \\
\noalign{and} \\
\omega &= k^{(s)}/(k^{(s)} + k^{(a)}).
\end{align}
For a Planckian source $j_{lm}(\bx,\bn)=k^{(a)} \sqrt{4\pi} B(\bx) \delta_{l0}
\delta_{m0}$ 
where $B(\bx)$ is isotropic.
The equation therefore becomes:
\begin{equation}
\begin{split}
c_{l-1,m}^+ { {dI_{l-1,m}(\tau)} \over {d\tau} } &+
c_{l+1,m}^- { {dI_{l+1,m}(\tau)} \over {d\tau} } = \\
& s_l I_{lm}(\tau) -s_0 \sqrt{4\pi} B(\tau) \delta_{l0} \delta_{m0} \\
&-\omega g_l Y_l^{m*} (\bn_\odot) I_\odot (\tau)
\end{split}
\end{equation}
where $s_l=1-\omega g_l$. For conservative scattering $s_0=0$, which case
will require some special treatment.
To solve these equations we divide the atmosphere
into $N$ homogeneous layers with optical thicknesses $\tau_i, i=1,\ldots, N$
and boundaries at optical depths $\Delta_i, i=0,\ldots, N$ 
in each of which the optical properties are
constant: $\tau$ will be used as a local optical depth when considering a 
single layer.
As these equations are linear the solution is the sum of a particular
integral and a complementary function.

\subsection{The Complementary Function}

Since the equation is linear the complementary function will consist of
a sum of exponentials of the form $H_{lm}(\mu)e^{\tau/\mu}$ 
for $\mu \in \mathbb{R}$. 
For any value of $\mu$ and a fixed value of $m$, a recurrence relation
may be established for the coefficients $H_{lm}$, starting from $H_{mm}$.
The expansion of the radiance in spherical harmonics is truncated at an
odd order $L$, so this recurrence must terminate with $H_{L'+1,m}=0$ where
$L'=L$ if $m$ is even and $L'=L+1$ if $m$ is odd (the reason for this 
is explained below). This
imposes a constraint on the permissible values of $\mu$ and defines an
eigenvalue problem.
\begin{align}
&c_{m+1,m}^- H_{m+1,m} = s_m \mu H_{mm},  \\
&c_{l-1,m}^+ H_{l-1,m}+ c_{l+1,m}^- H_{l+1,m} 
= s_l \mu H_{lm}, \quad m < l < L', \\
\noalign{and}
& c_{L'-1,m}^+ H_{L'-1, m} =  s_{L'} \mu H_{L'm}
\end{align}
This may be cast in a more usual form by defining $K_{lm}=\sqrt{s_l} H_{lm}$
so that
\begin{equation}
\sum_{l=m}^L C_{ql} K_{lm} =\mu K_{lm}, \quad m \leqslant q \leqslant L',
\end{equation}
where the non-zero entries in the matrix $C$ are given by:
\begin{equation}
C_{l-1,l}=c_{l-1,m}^+/\sqrt{s_l s_{l-1}} \quad \text{and}\quad
C_{l,l+1}=c_{l+1,m}^-/\sqrt{s_l s_{l+1}},
\end{equation}
where $m \leqslant l \leqslant L'$. In fact, since $c_{lm}^+=c_{l+1,m}^-$ the matrix $C$
is symmetrical. As it is also tridiagonal, the eigenvalues could be found
directly using the QR-algorithm, though it is possible to reduce the size 
of the problem as discussed below. Once the eigenvalues have been determined
the recurrence relation may be used to determine the $K_{lm}$. 

Care is needed with the recurrence. As $l \rightarrow \infty$ $c_{l,m}^\pm
\sim 1/2$ and $s_l \sim 1$. Hence, the recurrence approaches the form
\begin{equation}
H_{l-1,m}+ H_{l+1,m} =  2 \mu H_{lm}, 
\end{equation}
When $|\mu|> 1$ this has growing solutions, which will be triggered by
rounding errors in numerical practice. Physically, we seek a solution
which decays as $l\rightarrow \infty$, so the recurrence must be used 
in the direction of decreasing $l$, in which direction the desired solution
grows and will swamp the error. When $|\mu|<1$ recurrence in the
upward or downward direction is stable, so for algorithmic convenience 
downward recurrence is used consistently. (Note that \cite{Benassi84} 
use upward recurrence in this case, but it is not necessary to do so).
One further refinement is required in practice. When scattering is 
almost conservative, one eigenvalue is very large and traversing the
sequence in the downward direction terms increase by a factor of about
$2\mu$ at each stage. When the order of truncation is large enough
this can lead to numerical overflows. The recurrence itself is therefore 
recast in the quantities $H_{lm}'=\sigma^{-l} H_{lm}$, where $\sigma=
1/\max(1, 2\mu-1)$, to separate the overflowing behaviour while 
not affecting behaviour for small values of $\mu$.

({\it Note:} For comparison with the program we can define 
\begin{equation}
E_j=c_{m-2+j,m}^+/\sqrt{s_{m-1+j}s_{m-2+j}} \quad 2 \leqslant j \leqslant L+1-m
\end{equation}
as the subdiagonal element on the $j$th row of the matrix. Because the
optical properties of the layer do not depend on direction we might expect
that if $e^{\tau/\mu}$ is an eigensolution, $e^{-\tau/\mu}$ should also be.
This is seen to be so by observing that if $\mu$ is an eigenvalue with an
eigenvector $K_{lm}(\mu)$, a vector for which every other element
of $K_{lm}(\mu)$ is changed in sign will be an eigenvector for an eigenvalue
$-\mu$ as $C_{ij}=0$ unless $|i-j|=1$. This explains why odd and even orders
are truncated separately: if the eigenproblem is of an odd size $\mu=0$ will
be an eigenvalue, causing numerical overflows in evaluating the exponential.
Writing the eigenvector for the eigenvalue $\mu$ as $\mathbf{K}_e
+\mathbf{K}_o$, where the first term contains the even entries and the second
the odd entries, it follows that
\begin{align}
C (\mathbf{K}_e+\mathbf{K}_o)&= \mu (\mathbf{K}_e+\mathbf{K}_o) \\
\noalign{and}
C (\mathbf{K}_e-\mathbf{K}_o)&= -\mu (\mathbf{K}_e-\mathbf{K}_o)
\end{align}
from which
\begin{align}
C \mathbf{K}_e &= \mu \mathbf{K}_o \\
\noalign{and}
C \mathbf{K}_o &= \mu \mathbf{K}_e
\end{align}
so that
\begin{equation}
C^2 \mathbf{K}_o = \mu^2 \mathbf{K}_o
\end{equation}
By direct calculation the $(C^2)_{ij}=0$ if $i-j$ is odd. This means that even
rows and columns can be deleted from $C^2$ to halve the size of the 
eigenproblem. Indexing the rows of {\em this} matrix with $j$ and denoting 
the main diagonal elements by $d_j$ and the sub-diagonal elements by $e_j$,
\begin{align}
d_j &= E_{2j-1}^2 + E_{2j}^2 \\
\noalign{and}
e_j &= E_{2j-2} E_{2j-1}
\end{align}
for $1\leqslant j \leqslant (L'+1-m)/2$: here $E_1=0$)

The eigenvalues are of the form $\pm \mu_k$, $k=1,\ldots(L'+1-m)/2$, 
so the complementary function may be written as
\begin{equation}
I_{lm}(\tau) = \sum_k H_{lmk}^- e^{-\tau/\mu_k} + \sum_k H_{lmk}^+ 
e^{-(\tau_i-\tau)/\mu_k}
\end{equation}
where we follow Stamnes {et al.} in using only negative exponentials so as 
to avoid overflows when $\tau_i$ is large. The coefficients $H_{lmk}^\pm$
are determined by the eigenvectors, $\mathbf{K}_k$ of the matrix $C$. In fact,
\begin{equation}
H_{lmk}^\pm = u_{mk}^\pm s_l^{-1/2} (\pm 1)^{(m+l)} K_{klm}
\end{equation}
It is now convenient to define $\mathbf{V}_k$ so that $V_{klm}=K_{klm}/
\sqrt{s_l}$.

Conservative scattering poses a certain difficulty. As $\omega \rightarrow 1$,
the matrix $\mathbf{C}$ becomes singular in the case where $m=0$. Then, 
$\mathbf{C}$ has two eigenvalues of $O((1-\omega)^{-1/2})$ with eigenvalues
$\mathbf{K}=(1,\pm 1, 0, \ldots, 0) + O((1-\omega)^{-1/2})$ and eigenvalues
of $O(1)$ with eigenvectors $\mathbf{K}= (O((1-\omega)^{-1/2}), 
O((1-\omega)^{-1/2}), O(1), \ldots, O(1))$. When $\omega=1$ a solution
linear in $\tau$ must be sought. Since we may want to solve for a number
of atmospheric columns simultaneously it is desirable to avoid special 
pleading for singular cases, so for the present we artificially reduce
$\omega$ to avoid ill-conditioning: this seems to preform well enough
in practice, but it may be undesirable in extremely optically thick
conservative layers.

\subsection{The Particular Integral for Thermal Radiation}

In the infra-red region is is most convenient to reformulate the equation
of transfer in terms of differential radiances. We write
\begin{equation}
I=I'+B
\end{equation}
so that the equation of transfer becomes
\begin{equation}
\begin{split}
(\bn . \nabla) I'(\bx, \bn) & = -(k^{(s)} + k^{(a)}) I' (\bx, \bn) \\
& + { k^{(s)} \over
{4 \pi} } \int_{\Omega} I'(\bx,\bnp) P(\bnp, \bn) \, d\omega_{\bnp} 
- (\bn . \nabla) B(\bx).
\end{split}
\end{equation}
Introducing the optical depth, $\tau$
\begin{equation}
\begin{split}
n_0 {{d I'(\tau, \bn)} \over {d\tau} }& = I' (\tau, \bn) 
 - {\omega \over
{4 \pi} }  \int_{\Omega} I'(\bx,\bnp) P(\bnp, \bn) \, d\omega_{\bnp} 
- n_0 {{d B(\tau)}\over {d\tau}}
\end{split}
\end{equation}
Now, $n_0=\sqrt{4\pi/3}. Y_1^0(\bn)$, so on expanding this in spherical
harmonics,
\begin{equation}
\begin{split}
\sum_{lm} n_0 Y_l^m (\bn) { {dI'_{lm} (\tau)} \over {d\tau} } &= 
\sum_{lm} I'_{lm}(z) Y_l^m (\bn) \\
&- \omega \sum_{lm} I'_{lm}(z) g_l Y_l^m (\bn) \\
&- \sqrt{4\pi/3} \delta_{l1} \delta_{m0} Y_l^m (\bn) 
{{d B(\tau)}\over {d\tau}}
\end{split}
\end{equation}
Proceeding as before,
\begin{equation}
\begin{split}
c_{l-1,m}^+ { {dI_{l-1,m}(\tau)} \over {d\tau} } &+
c_{l+1,m}^- { {dI_{l+1,m}(\tau)} \over {d\tau} } = \\
& s_l I_{lm}(\tau) - \sqrt{4\pi/3} \delta_{l1} \delta_{m0} 
{{d B(\tau)}\over {d\tau}}.
\end{split}
\end{equation}
The simplest case to consider is that when $B$ is linear in $\tau$. The 
particular integral then becomes
\begin{equation}
I_{i, lm}={1\over s_{1i}} \sqrt{{4\pi}\over 3} \, {{\Delta B_i}\over\tau_i}
\delta_{l1}\delta_{m0}
\end{equation}
where $\Delta B_i$ is the difference in the Planckian across the $i$th
layer in the direction of increasing $\tau$.

We also consider the case where the variation of the Planckian is
quadratic across the layer. In this case we have
\begin{align}
I_{i,10} &= {1\over s_{1i}} \sqrt{{4\pi}\over 3} \, 
{{\Delta B_i}\over\tau_i} - {2\over s_{1i}} \sqrt{{4\pi}\over 3} 
{{\Delta^2 B_i}\over\tau_i^2} \tau\\
I_{i,00} &= - {2 c_{1,0}^-\over {s_{0i} s_{1i}}} \sqrt{{4\pi}\over 3}
{{\Delta^2 B_i}\over\tau_i^2}\\
\noalign{and}
I_{i,20} &= - {2 c_{1,0}^+ \over {s_{2i} s_{1i}}} \sqrt{{4\pi}\over 3}
{{\Delta^2 B_i}\over\tau_i^2}
\end{align}
with $I_{i,lm}=0$ otherwise.

\subsubsection{Small Optical Depths}

The solutions will clearly fail in the case when $\tau=0$, but even when
$\tau$ is not quite 0 ill-conditioning will arise; this could theoretically
be overcome by increasing $\tau$ to some mimimum value, but in practice
such a value would be unacceptably large. Conditioning is therefore 
restored by adding to the particular integral a solution of the homogeneous
system which exhibits the same singularity as $\tau\rightarrow 0$. We 
consider only the case of linear variations in $\tau$ for now. Restricting
ourselevs to the relevant case $m=0$ the foregoing particular integral can
be written as
\begin{equation}
I_{l0}=q_{0} \delta_{l1},
\end{equation}
where $q_{0}$ is a constant. As the optical depth tends to 0, the homogeneous
solution becomes
\begin{equation}
I_{l0}=\sum_k \left \{ u_k^+ V_{kl} + u_k^- (-1)^l V_{kl} \right \}
+ O(\tau/\mu_k).
\end{equation}
Since $C$ is real and symmetric its eigenvectors, ${\bf K}_k$, are orthogonal
and may be normalized. We therefore have
\begin{align}
\sum_l V_{kl} V_{k'l} s_l &= \delta_{kk'} \\
\noalign{and} 
\sum_l V_{kl} V_{k'l} (-1)^l s_l &= 0 
\end{align}
We immediately find that
\begin{align}
u_k^+ &= -q_0 s_1 V_{k1} \\
\noalign{and} 
u_k^- &= q_0 s_1 V_{k1} 
\end{align}
so the homogeneous solution to restore conditioning becomes
\begin{equation}
I_{l0}=q_0 s_1 \sum_k V_{kl}V_{k1} \left \{ (-1)^l e^{-\tau/\mu_k} 
-e^{-(\tau_i-\tau)/\mu_k} \right \} .
\end{equation}


\subsection{The Solar Particular Integral}

Using the standard notation $\mu_0 = - \cos \theta_\odot$ the direct solar beam
in a layer may be written as
\begin{equation}
I_{\odot i}(\tau) = I_\odot (\Delta_{i-1}) e^{-\tau/\mu_0}
\end{equation}
Provided that $\mu_0 \neq \mu_k$ for any eigenvalue $\mu_k$ a particular 
integral of the form $I_{ilm}(\tau)=Z_{ilm} e^{-\tau/\mu_0}$ may be sought.
This gives
\begin{equation}
c_{l-1,m}^+ Z_{i,l-1,m} + c_{l+1,m}^- Z_{i,l+1,m}
= -\mu_0 s_{li} Z_{ilm} +\mu_0 I_\odot (\Delta_{i-1})
\omega_i g_{li} Y_l^{m*} ( \bn_\odot).
\end{equation}
A truncation is imposed by setting $Z_{i,L'+1,m}=0$. Noting that $\omega g_l
=1-s_l$ and that $\mu_0 = -(\bn_\odot)_0$, it follows on using the
recurrence relation that
\begin{equation}
\begin{split}
&c_{l-1,m}^+ (Z_{i,l-1,m} + I_\odot (\Delta_{i-1}) Y_{l-1}^{m*} ( \bn_\odot))
\\
&+ c_{l+1,m}^- (Z_{i,l+1,m}+ I_\odot (\Delta_{i-1}) Y_{l+1}^{m*}( \bn_\odot))
\\
&= -\mu_0 s_{li} (Z_{ilm} +I_\odot (\Delta_{i-1}) Y_l^{m*} ( \bn_\odot))
\end{split}
\end{equation}
This admits a solution
\begin{equation}
Z_{ilm} = - I_\odot (\Delta_{i-1}) Y_l^{m*} (\bn_\odot) 
+ \gamma \mathcal{V}_{ilm}(\mu_0)
\end{equation}
with
\begin{equation}
\gamma= I_\odot (\Delta_{i-1}) Y_{L'+1}^{m*} ( \bn_\odot)/ 
\mathcal{V}_{i,L'+1,m} (\mu_0)
\end{equation}
where $\mathcal{V}(\mu_0)$ is defined by the recurrence
\begin{equation}
c_{l-1,m}^+ \mathcal{V}_{i,l-1,m} + c_{l+1,m}^- \mathcal{V}_{i,l+1,m} 
= - \mu_0 s_l \mathcal{V}_{ilm}
\end{equation}
starting from $\mathcal{V}_{imm}=1$.

The issue of ill-conditioning must be addressed here. If $\mu_0$ is close
to one of the eigenvalues of the linear system ill-conditioning will
arise, with a singularity in the case when equality obtains. This can be 
removed by finding the eigenvalue closest to $\mu_0$ and subtracting from the 
particular integral a multiple of the coressponding eigensolution which
cancels the singularity in the limit. Instead of implementing this using
an {\tt IF}-test, it is applied using a weighting involving the separation
of $\mu_0$ and the eigenvalue and so removes ill-conditioning at nearby values.

\subsection{Interior Boundary Conditions}

On interior boundaries we must apply the conditions
\begin{equation}
I_{ilm}(\tau_i) =I_{i+1,lm} (0), \quad 1\leqslant i \leqslant N, \forall l,m.
\end{equation}
We write the particular integral in the $i$th layer as $\hat G_{ilm}$ at the 
top and as $\check G_{ilm}$ at the bottom. Then,
\begin{equation}
\begin{split}
0&=\sum_k \biggl \{ u_{mik}^- (-1)^{l+m} V_{lmik} \vartheta_{ik} 
+ u_{mik}^+ V_{lmik} + \check G_{lmi} \\
& -u_{m,i+1,k}^- (-1)^{l+m} V_{lm,i+1,k} 
- u_{m,i+1,k}^+ V_{lm,i+1,k} \vartheta_{i+1,k} - 
\hat G_{lm,i+1} \biggr \}
\end{split}
\end{equation}
for $l=m,\ldots,L'$.

\subsection{The Upper boundary Condition}

At the top boundary of the atmosphere the radiance must be specified in
downward directions. Typically, the indicent radiation will comprise
only the direct solar beam, but we shall formulate the boundary condition
more generally to allow for possiblities such as the use of 
differential radiances in the infra-red. The condition is then
\begin{equation}
I(\bn)= I^{(0)} (\bn), \qquad \bn \in \Omega_-.
\end{equation}
where $I^{(0)}=\sum_{lm} I_{lm}^{(0)} Y_l^m(\bn)$.
As $I^{(0)}$ is specified only on
$\Omega_-$, the coeffieicnts $I_{lm}^{(0)}$ are not uniquely defined, but
they can be made so by making $I^{(0)}$ symmetric or antisymmetric.

In a truncated system it is not possible to impose the boundary condition
for every $\bn \in \Omega_-$. The simplest possibility is to specify 
that $I(\bn)=I^{(0)}(\bn)$ for a finite number of
$\bn$, but most authors prefer Marshak's conditions
\begin{equation}
\int_{\Omega_-} (I(\bn)-I^{(0)}(\bn)) Y_{l'}^{m'*} (\bn) \, d \omega_{\bn} =0
\end{equation}
for those $Y_{l'}^{m'}$ with odd parity. The equation becomes trivial if
$m' \neq m$, so considering a fixed value of $m$,
this restricts us to $l'=m+1, \ldots, L'$. The boundary conditions are
therefore
\begin{equation}
\sum_{l} \kappa_{ll'm}  (I_{lm}-I_{lm}^{(0)})=0
\end{equation}
for the given $l'$, where,
\begin{equation}
\kappa_{ll'm}=\int_{\Omega_-} Y_l^m(\bn) Y_{l'}^{m*} (\bn) \, d\omega_{\bn}.
\end{equation}
Substituting the expression for $I_{lm}$ we obtain the equation
\begin{equation}
\begin{split}
\sum_l \kappa_{ll'm} (I_{lm}^{(0)}-\hat G_{lm1}) &=
\sum_k \Bigg \{ u_{m1k}^- \left ( \sum_l \kappa_{ll'm} V_{lm1k} (-1)^{l+m}
\right ) \\
&+ u_{m1k}^+ \left ( \sum_l \kappa_{ll'm} V_{lm1k}
\right ) \vartheta_{1k} \Bigg \}
\end{split}
\end{equation}
Turning to the calculation of $\kappa_{ll'm}$ note that
\begin{equation}
\begin{split}
\int_{\Omega_-} Y_l^m(\bn) Y_{l'}^{m'*} (\bn) \, d\omega_{\bn}
&= \int_{\Omega_+} Y_l^m(-\bn) Y_{l'}^{m'*} (-\bn) \, d\omega_{\bn} \\
&= (-1)^{l+m+l'+m'} \int_{\Omega_+} Y_l^m(\bn) Y_{l'}^{m'*} (\bn) 
\, d\omega_{\bn} 
\end{split}
\end{equation}
A number of simplifications may now be made. If $l+l'$ is even, the integrand
is even and will have the same value on $\Omega_+$, so extending the integral
and applying orthogonality,
\begin{equation}
\kappa_{ll'm}= 1/2 \; \delta_{ll'}
\end{equation}
if $l+l'$ is even.

If $l+l'$ is odd, the evaluation of $\kappa_{ll'm}$ is not so trivial. 
\cite{Dave74} give results for the case $m=0$. To derive the more general
results required here, it seems easiest to follow the procedure given in
\cite{Copson} for Legendre polynomials and proceed from the basic 
differential equation. Defining
\begin{equation}
Y_l^m \equiv \Upsilon_l^m e^{im\phi} \equiv \Xi_l^m P_l^m e^{im\phi},
\end{equation}
it follows that
\begin{equation}
\int_{\Omega_+} Y_l^m Y_{l'}^{m*} \, d\omega_{\bn} =
2\pi \, \Xi_l^m \Xi_{l'}^m \, \int_0^1 P_l^m(x) P_{l'}^m(x) \, dx
\end{equation}
By definition,
\begin{equation}
{d\over{dx}} \left [ (1-x^2) {{dP_l^m}\over{dx}} \right ] +
\left [ l(l+1) - {m^2 \over {1-x^2}} \right ] P_l^m =0.
\end{equation}
Multiplying by $P_{l'}^m$, subtracting $P_l^m$ multiplied by the
corresponding differential equation for $P_{l'}^m$, and integrating
by parts,
\begin{equation}
\begin{split}
(l-l')(l+l'+1) P_l^m P_{l'}^m &= {d\over{dx}} \left [
P_l^m (1-x^2) {{dP_{l'}^m}\over{dx}} \right ] - (1-x^2) {{dP_l^m}\over{dx}}
{{dP_{l'}^m}\over{dx}} \\
&- {d\over{dx}} \left [P_{l'}^m (1-x^2) {{dP_l^m}\over{dx}} \right ]
+(1-x^2) {{dP_{l'}^m}\over{dx}} {{dP_l^m}\over{dx}}.
\end{split}
\end{equation}
Hence,
\begin{equation}
\int_0^1 P_l^m P_{l'}^m \, dx = {{  (1-x^2) \left \{ P_l^m { { dP_{l'}^m}
\over {dx} } - P_{l'}^m { { dP_l^m}\over {dx} } \right \} \biggm |_0^1}
\over {(l-l')(l+l'+1)}}
\end{equation}
Only the lower limit gives a contribution. To evalaute this note that when
$x$ is small
\begin{equation}
P_l^m (x) \sim {{(-1)^{m+l}}\over {2^l l!}} \left [ 1 -{m\over 2} x^2 +
\ldots \right ] \, {d^{l+m}\over{dx^{l+m}}} \sum_{r=0}^l 
\left ( {l \atop r} \right ) (-1)^r x^{2r}
\end{equation}
When $x=0$ the only contribution arises from the term of the final series
with $2r=l+m$, so $l+m$ must be even. 
\begin{equation}
\therefore\qquad P_l^m (0) = { { (-1)^{{m+l}\over 2} } \over { 2^l l!}} \;
{ {(l+m)!} \over { \left ( {{l+m}\over 2} \right )! \left ( {{l-m}\over 2} 
\right )! } }.
\end{equation}
Similarly, the only contribution to $dP_l^m/dx$ arises
from the term with $2r=l+m+1$, so $l+m$ must be odd.
\begin{equation}
\therefore\qquad {{dP_l^m (0)}\over{dx}} = { { (-1)^{{m+l-1}\over 2} } 
\over { 2^l l!} } \;
{ {(l+m+1)!} \over { \left ( {{l+m+1}\over 2} \right )! 
\left ( {{l-m-1}\over 2} \right )! } }.
\end{equation}

From a numerical point of view, these are easiest to evaluate using
recurrences:
\begin{equation}
P_l^m(0) = - {{l+m-1}\over{l-m}} \, P_{l-2}^m (0)
\end{equation}
with $P_m^m(0)=(-1)^m/2^m . (2m)! /m!$ when $l+m$ is even and
\begin{equation}
{ {d P_l^m(0)} \over {dx}} = - {{l+m}\over{l-m-1}} \, {{dP_{l-2}^m (0)} 
\over {dx}}
\end{equation}
with $P_{m+1}^m(0)=(-1)^m/2^{m+1}. (2(m+1))!/(m+1)!$ when $l+m$ is odd.
Finally, it is useful to express these in terms of $\Upsilon_l^m$ to keep
terms closer to 1:
\begin{equation}
\Upsilon_l^m(0) = -\sqrt {{{(2l+1)}\over{(2l-3)}} . {{(l+m-1)}\over{(l+m)}} .
{{(l-m-1)}\over{(l-m)}}} 
\,\Upsilon_{l-2}^m (0)
\end{equation}
with $\Upsilon_m^m(0)=(-1)^m/2^m \cdot 1/m! \cdot \sqrt {(2m+1)! / {4\pi}}$ 
when $l+m$ is even and
\begin{equation}
{ {d \Upsilon_l^m(0)} \over {dx}} = -\sqrt {{{(2l+1)}\over{(2l-3)}} .
{{(l-m)}\over{(l-m-1)}} .{{(l+m)}\over{(l+m-1)}}}
\, {{d\Upsilon_{l-2}^m (0)}
\over {dx}}
\end{equation}
with $\Upsilon_{m+1}^m(0)=(-1)^m/2^m . 1/m! . \sqrt {(2m+3).
(2m+1)! / {4\pi}}$ when $l+m$ is odd.

Finally, therefore,
\begin{equation}
\begin{split}
\kappa_{ll'm} &= \int_{\Omega_-} Y_l^m (\bn) Y_{l'}^{m'*} (\bn) \, d\omega_\bn
= (-1)^{l+l'} \delta_{mm'} \int_{\Omega_+} \Upsilon_l^m (\bn) 
\Upsilon_{l'}^{m} (\bn) \, d\omega_\bn \\
&= 2\pi (-1)^{(l+l'+1)} { {\Upsilon_l^m (0){d\Upsilon_{l'}^m (0)/dx} 
- \Upsilon_{l'}^m (0){d\Upsilon_l^m (0)/dx}}\over{(l-l')(l+l'+1)}}
\end{split}
\end{equation}

({\em Note:} For comparison with the program $\Upsilon_l^m(0)=0$ if 
$d\Upsilon_l^m(0)/dx \not = 0$, so only one array is required to hold
both quantities. Also, only one term in the numerator of the preceeding 
equation can be non-zero.)

\section{Boundary Conditions at the Surface}

To define the surface characteristics we must use a bidirectional reflectance,
function $\gamma_r$, so that the reflected ray in the direction $\bn \in 
\Omega_+$, is given by
\begin{equation}
I(\bn)=\int_{\Omega_-} \gamma_r(\bn,\bnp) I(\bnp) (\bnp . -\mathbf{e}_z) \,
d\omega_{\bnp}
\end{equation}
where the geometrical factor $\bnp . -\mathbf{e}_z$ accounts for the
projected area of the horizontal surface seen by the incident beam.
In the case of a Lambertian surface $\gamma_r$ is a constant and may be 
related to the albedo of the surface by $\gamma_r=\alpha/\pi$, which follows 
directly from the definition. (For scattering into finite solid angles a
biconical reflectance is defined as
\begin{equation}
R(\bn, \bnp)= { {\int_{\Omega_r} \int_{\Omega_i} \gamma_r(\bn,\bnp) I(\bnp)
(\bnp . -\mathbf{e}_z) (\bn . \mathbf{e}_z) \, d\omega_{\bnp} 
d\omega_{\bn} } \over { \int_{\Omega_r} \int_{\Omega_i} {1\over \pi} I(\bnp)
(\bnp . -\mathbf{e}_z) (\bn . \mathbf{e}_z) \, d\omega_{\bnp}
d\omega_{\bn} } }
\end{equation}
where the factor of $1/\pi$ in the denominator represents the BRDF of a
white Lambertian surface.)

	For use in a spherical harmonic procdure, the BRDF may be expanded
in a double spherical harmonic series:
\begin{equation}
\gamma_r(\bn,\bnp) = \sum_{l,m} \sum_{l',m'} \Gamma_{lml'm'} Y_l^m(\bn)
Y_{l'}^{m'*} (\bnp)
\end{equation}
where the use of complex conjugates in the second sum is for convenience.
Various constraints on the coefficients $\Gamma_{lml'm'}$ must be
imposed, limiting the number of free coefficients. Firstly, $ \gamma_r \in
\mathbb{R}$ so
\begin{equation}
\begin{split}
\sum_{l,m} \sum_{l',m'} \Gamma_{lml'm'} &Y_l^m(\bn) Y_{l'}^{m'*} (\bnp) 
=\sum_{l,m} \sum_{l',m'} \Gamma_{lml'm'}^* Y_l^{m*}(\bn) Y_{l'}^{m'} (\bnp) \\
&=\sum_{l,m} \sum_{l',m'} \Gamma_{lml'm'}^* (-1)^m Y_l^{-m}(\bn) (-1)^{m'} 
Y_{l'}^{-m'*} (\bnp) \\
&=\sum_{l,m} \sum_{l',m'} \Gamma_{l,-m,l',-m'}^* (-1)^{(m+m')} Y_l^{m}(\bn) 
Y_{l'}^{m'*} (\bnp)
\end{split}
\end{equation}
Hence,
\begin{equation}
\Gamma_{l,-m,l',-m'}=(-1)^{(m+m')} \Gamma_{lml'm'}^*.
\end{equation}
Helmholtz's principal of reciprocity imposes a requirement that
\begin{equation}
\gamma_r(\bn,\bnp) = \gamma_r(\bnp,\bn);
\end{equation}
hence,
\begin{equation}
\begin{split}
\sum_{l,m} \sum_{l',m'} \Gamma_{lml'm'} &Y_l^m(\bn) Y_{l'}^{m'*} (\bnp) 
=\sum_{l,m} \sum_{l',m'} \Gamma_{lml'm'} Y_l^m(\bnp) Y_{l'}^{m'*} (\bn) \\
&=\sum_{l',m'} \sum_{l,m} \Gamma_{l'm'lm} Y_{l'}^{m'}(\bnp) Y_l^{m*} (\bn) \\
&=\sum_{l',m'} \sum_{l,m} \Gamma_{l'm'lm} (-1)^{(m+m')} Y_l^{-m}(\bn) 
Y_{l'}^{-m'*} (\bnp) \\
&=\sum_{l',m'} \sum_{l,m} \Gamma_{l',-m',l,-m} (-1)^{(m+m')} Y_l^m(\bn) 
Y_{l'}^{m'*} (\bnp)
\end{split}
\end{equation}
whence,
\begin{equation}
\Gamma_{l',-m,l,-m'}=(-1)^{(m+m')} \Gamma_{lml'm'}.
\end{equation}
In addition to these general properties we impose the specific constraints
of rotational and reflectional symmetry:
\begin{align}
\gamma_r(\mathcal{R}(\bn),\mathcal{R}(\bnp)) &= \gamma_r(\bnp,\bn) \\
\noalign{and}
\gamma_r(\mathcal{I}(\bn),\mathcal{I}(\bnp)) &= \gamma_r(\bnp,\bn) \\
\end{align}
for any rotation $\mathcal{R}$ about a vertical axis and any inversion 
$\mathcal{I}$ in a vertical plane. Since
\begin{equation}
Y_l^m(\mathcal{R}(\bn))=e^{im\phi_{\mathcal{R}}} Y_l^m(\bn),
\end{equation}
we have,
\begin{equation}
\begin{split}
\gamma_r(\mathcal{R}(\bn),\mathcal{R}(\bnp)) &= \sum_{l,m} \sum_{l',m'} 
\Gamma_{lml'm'} Y_l^m(\mathcal{R}(\bn)) Y_{l'}^{m'*} (\mathcal{R}(\bnp)) \\
&=\sum_{l,m} \sum_{l',m'} \Gamma_{lml'm'} e^{im\phi_{\mathcal{R}}} 
Y_l^m(\bn) e^{-im'\phi_{\mathcal{R}}} Y_{l'}^{m'*} (\bn) \\
&=\sum_{l,m} \sum_{l',m'} \Gamma_{lml'm'} Y_{l}^{m}(\bn) Y_l^{m*} (\bnp) 
\end{split}
\end{equation}
which can be true only if
\begin{equation}
\Gamma_{lml'm'}=\Psi_{ll'm}\delta_{mm'}
\end{equation}
for suitable $\Psi$. Now, to impose reflectional symmetry, it suffices to
consider inversion in the plane $\phi=0$:
\begin{equation}
\begin{split}
\gamma_r(\mathcal{I}(\bn),\mathcal{I}(\bnp)) &= \sum_{l,m} \sum_{l',m'}
\Gamma_{lml'm'} Y_l^m(\mathcal{I}(\bn)) Y_{l'}^{m'*} (\mathcal{I}(\bnp)) \\
&=\sum_{l,m} \sum_{l',m'} \Gamma_{lml'm'} Y_l^{m*}(\bn) Y_{l'}^{m'} (\bnp) \\
&=\sum_{l,m} \sum_{l',m'} \Gamma_{l,-m,l',-m'} (-1)^{(m+m')} 
Y_{l}^{m}(\bn) Y_{l'}^{m'*} (\bnp)
\end{split}
\end{equation}
from which it follows that
\begin{equation}
\Gamma_{l,-m,l',-m'}=(-1)^{(m+m')} \Gamma_{lml'm'}.
\end{equation}
Together with the condition the imposed by $\gamma_r \in \mathbb{R}$, this
shows that $\Gamma_{lml'm'}\in \mathbb{R}$.

Collecting these results, we find that
\begin{align}
\Psi_{ll'm} &\in \mathbb{R}; \\
\Psi_{l'lm} &= \Psi_{ll'm} \\
\noalign{and}
\Psi_{ll'-m} &= \Psi_{ll'm} 
\end{align}

Since the BRDF is defined only for $\bn \in \Omega_+$ and $\bnp \in \Omega_-$
the $\Psi{ll'm}$ are not uniquely defined. We can, hwoever, coplete the
specification by demanding that $\Psi_{ll'm}=0$ if $l=m$ or $l'+m$ is odd.
This is the natural choice since a Lambertian surface is then characterized
by one value of $\Psi$: namely that with $l=l'=m=0$.

\subsection{The Relation between the BRDF and the Albedo}

In some instances it is useful to know the relationship between the
BRDF and the albedo for isotropic incident radiation. This may be derived 
as follows.

\begin{equation}
\begin{split}
\alpha_i &= {1\over \pi} \int_{\Omega_+} \int_{\Omega_-} \gamma ( \bnp , \bn)
(-\bnp . \hat {\bf e}_z ) (\bn . \hat {\bf e}_z ) \, d\omega{\bn} \,
d\omega{\bnp} \\
&= {1\over \pi} \sum_{ll'm} \Psi_{ll'm} \int_{\Omega_+} \int_{\Omega_-}
Y_{l'}^m (\bnp) Y_l^m (\bn) (-\bnp . \hat {\bf e}_z ) (\bn . \hat {\bf e}_z ) 
\, d\omega{\bnp} \, d\omega{\bn} \\
&= {1\over \pi} \sum_{ll'm} \Psi_{ll'm} \int_{\Omega_+} Y_l^m (\bn)
(\bn . \hat {\bf e}_z ) \, d\omega{\bn}. (-1) \int_{\Omega_-} Y_{l'}^m (\bnp) 
(\bnp . \hat {\bf e}_z) \, d\omega{\bnp} \\
&= {1\over \pi} \sum_{ll'm} \Psi_{ll'm} (-1)^{l+m} \int_{\Omega_+} Y_l^m (\bn)
\sqrt{{{4\pi}\over 3}} Y_1^0(\bn) \, d\omega{\bn} \int_{\Omega_-}
Y_{l'}^m (\bnp) \sqrt{{{4\pi}\over 3}} Y_1^0(\bnp) \, d\omega{\bnp} \\
&= {4\over 3} \sum_{ll'} (-1)^l \Psi_{ll'} \kappa_{l10} \kappa_{l'10}.
\end{split}
\end{equation}




\subsection{Specification of Real BRDFs}

Various analytic expressions for BRDFs have been proposed. These typically
represnt a blend of physical reasoning and fitting to experimental data.
An example is provided by \cite{Roujean92} who considers the effect of 
geometric irregularities on the surface which produce shadowing effects
and of radiative transfer in the medium below the surface which is treated
by solving the equation of transfer with a highly truncated phase function.
The model gives a BRDF of the form
\begin{equation}
\gamma_r(\theta_s, \theta_v, \phi) = k_0 +k_1 f_1(\theta_s, \theta_v, \phi)
+k_2 f_2(\theta_s, \theta_v, \phi)
\end{equation}
where $k_0, \ldots k_2$ are fitted constants, $f_1$ and $f_2$ are prescribed
functions and $\theta_s$ and $\theta_v$ are the polar angle of incident
radiation and the viewing angle. $f_1$ and $f_2$ have the following forms:
\begin{align}
f_1(\theta_s, \theta_v, \phi) &= {1\over {2\pi}} \left [ (\pi-\phi) \cos \phi
+\sin \phi \right ] \tan \theta_s \tan \theta_v 
 - {1\over\pi} \Bigl ( \tan \theta_s \\
& + \tan \theta_v + \sqrt {\left \{
\tan^2 \theta_s + \tan^2 \theta_v -2 \tan \theta_s \tan \theta_v \cos \phi
\right\}} \Bigr ). \nonumber \\
\noalign{and}
f_2(\theta_s, \theta_v, \phi) &= {3\over{4\pi}} {1\over{\cos\theta_s +
\cos\theta_v}} \left [ \left ( {\pi\over2} - \xi \right ) \cos \xi
+ \sin \xi \right ] -{1\over 3}
\end{align}
where 
\begin{equation}
\cos \xi = \cos\theta_s \cos\theta_v + \sin\theta_s\sin\theta_v \cos\phi
\end{equation}
Legendre expansions for $f_1$ and $f_2$ can be precalculated, so this
model is fairly easy to implement: Roujean {\it et al.}'s paper gives 
coefficients for some land surfaces. This has a convenient functional
form consisting of a linear combination of angularly dependent functions.
To simplify the treatment of the surface it will be assumed that the
BRDF may be expanded in the form
\begin{equation}
\gamma_r(\bn, \bnp) = \sum_j \rho_j f_j(\bn,\bnp)
\end{equation}
where the $\rho_j$ are functions of the surface type and the functions
$f_j$ (not necessarily equal to those above are known). It is then 
possible to precalculate the expansion of each $f_j$ in spherical harmonics:
\begin{equation}
f_j(\bn,\bnp) = \sum_{ll'm} F_{jll'm} Y_l^m(\bn) Y_{l'}^{m*}(\bnp)
\end{equation}
so that
\begin{equation}
\Psi_{ll'm}= \sum_j \rho_j F_{jll'm}.
\end{equation}

\subsection{The Optical Properties of the Ocean Surface}

Perhaps frustratingly, there is apparently no directly applicable reference
which provides a BRDF of the ocean surface. To provide such an entity the
radiance code itself can be used to calculate the radiatiance in the
ocean, with special upper boundary conditions to deal with refraction at
the surface. In order to implement such a capability the optical properties
of the ocean must be specified; these are greatly influenced by particulate
matter -- indeed, this is the basis of ocean colour sensing -- and very
considerable variations occur. An extremely useful review of this
field is provided by \cite{Mobley94}: a very brief discussion of ocean
optics for use in the present version of the code,  based on this book, 
is now presented.

\subsubsection{The Optical Properties of Oceanic Waters}

Rayleigh scattering occurs in the oceans just as it does in the atmosphere
and is described by a phase function
\begin{equation}
P_w(\theta) = {3\over {4\pi (3+p)}} (1+p\cos^2 \theta)
\end{equation}
where $p$ is the polarization factor, which is taken as $0.835$. The scattering
coefficient (m${}^{-1}$) has the wavelength dependence
\begin{equation}
k_w^{(s)}(\lambda) = K_R (\lambda_0/\lambda)^{4.32}
\end{equation}
where $\lambda_0=550$nm and $K_R=0.93$ for pure water and $K_R=1.21$ for
sea water. The dependence on wavelength is steeper than $\lambda^{-4}$ 
because of the effect of dissolved ions on the refractive index. (Note: the
values given in Table~3.8 of \cite{Mobley94} do not exactly follow this
realationship, which is presumably only applicable locally in frequency
space). 

Scattering by particulate matter is much more important than Rayleigh
scattering in almost all waters. \cite{Petzold72} has investigated the
phase function in various waters: to some extent, particulate scattering
can be represented by a Henyey-Greenstein phase function with an asymmetry
factor of 0.924, though this does not capture the full forward peak. The
scattering coefficient (m${}^{-1}$) of particulates is often related to 
the concentration of chlorophyll, $C$ (mgm${}^{-3}$) using the fitted 
formula:
\begin{equation}
k_P^{(s)}= \left ( {{550} \over { \lambda \mbox{[nm]}}} \right ) 0.3 C^{0.62}
\end{equation}
In the UM oceanic waters are assumed to be of type IB in Jerlov's 
classification for radiative purposes; it would seem sensible to assume the
same here and thus to take $C \approx 0.1$ mgm${}^{-3}$.

Absorption by oceanic waters is complicated by the presence of various
dissolved organic compounds which can give the water a yellow tinge and
are therefore often referred to as  {\em yellow matter}. By making the
questionable assumption the concentration of yellow matter is correlated
with that of phytoplankton \cite{Prieur81} produced an expression for the
absorption coefficient (m${}^{-1}$) of oceanic water that was simplified by
\cite{Morel91} to give
\begin{equation}
k^{(a)} = \left ( k_w^{(a)}(\lambda) +0.66 a_c^{*'} (\lambda) C^{0.65}
\right ) \left [ 1 + 0.2 \exp (-0.014 (\lambda\mbox{[nm]}-440 )) \right ].
\end{equation}
Here, $k_w^{(a)}$ is the absorption coefficient of pure water and $a_c^{*'}$
is the dimensionless absorption coefficient of chlorophyll.

\subsubsection{Conditions at the Oceanic Surface}

A discussion of conditions at the oceanic surface is presented by \cite{Mobley94}
who discusses level surfaces and also explains how waves can be treated. The
influence of waves on the BRDF is not negligible, but we do not currently
include a representation of waves in the specification of the albedo in the
UM and inclusion of such effects is by no means simple. Moreover, most
current published work on reflection from the ocean surface (\cite{Morel93},
\cite{Morel95} and \cite{Yang97}) does not include such effects.

Transfer across the surface into the ocean is governed by Snell's Law:
\begin{equation}
\sin \theta_i = n \sin \theta_t
\end{equation}
where $n$ is the real part of the refractive index and is quite close to 1.34
for oceanic waters at frequencies of interest. For unpolarized light Fresnel's
formulae may be combined to give an overall reflection coefficient:
\begin{equation}
r_{aw}= {1\over 2} \left \{ \left [ { {\sin (\theta_i -\theta_t) } \over
{\sin (\theta_i +\theta_t)} } \right ]^2 
+ \left [ { {\tan (\theta_i -\theta_t) } \over
{\tan (\theta_i +\theta_t)} } \right ]^2 \right \}.
\end{equation}
The radiance of the transmitted ray is then obtained from the fundamental 
theorem of radiometry as
\begin{equation}
I_t= n^2 (1-r) I_i.
\end{equation}
For rays travelling upward in the ocean similar considerations apply, but
with $n$ replaced by $1/n$. Principally, however, we are concerned with the
reflection coefficient in the water $r_{wa}$: for glancing incidence total
internal reflection occurs and $r_{wa}=1$, but generally it is given by
Fresnal's formula. The appropriate boundary condition is
\begin{equation}
I(\bn)=r_{wa} I(\bn_r) + (1-r_{aw}) n^2 I_a(\bn_a), \qquad \bn \in \Omega_-.
\end{equation}
where $\bn_r$ is the direction which is reflected to $\bn$ in the water and
$\bn_a$ is the direction in the air which is refrected to $\bn$. For the
purposes of determining a BRDF, we need consider only
\begin{equation}
I_a(\bn)=\delta(\bn-\bn_0).
\end{equation}
The effect of reflection is to cahnge the polar angle $\theta$ to $\pi-\theta$,
so since 
\begin{equation}
I=\sum_{lm} I_{lm} Y_l^m (\bn),
\end{equation}
we have
\begin{equation}
I(\bn_r)=\sum_{lm} I_{lm} Y_l^m(\bn_r) 
        = \sum_{lm} I_{lm} (-1)^{l+m} Y_l^m(\bn)
\end{equation}
Fresnel's coefficient $r_{wa}$ is axially symmetric so it may be written as
\begin{equation}
r_{wa}=\sum_\lambda \rho_\lambda Y_\lambda^0(\bn).
\end{equation}
Since the boundary condition applies only on $\Omega_-$, we must apply 
Marshak's procedure and form the inner product with $Y_L^M$ for those
spherical harmonics with odd parity. This leads to the condition
\begin{equation}
\sum_l \kappa_{LlM} I_{lM} = \sum_{l\lambda} r_\lambda I_{lM} C_{lM\lambda 0}
^{LM} + \left [ 1 - r_{aw}(\bn_0) \right ] n^2 Y_L^{M*}(\bnp_0).
\end{equation}
where $\bnp_0$ is the direction into which $\bn$ is refracted on entering the
ocean and $C_{lM\lambda 0}^{LM}$ is the Clebsch-Gordan coefficient. Note here
the general expression for the Clebsch-Gordan coefficient (\cite{Brink})
\begin{equation}
\begin{split}
C_{a\alpha b\beta}^{c\gamma} &= \delta(\alpha+\beta,\gamma) \Delta(a,b,c) \\
&\times \left [ (2c+1) (a+\alpha)!(a -\alpha)! (b+\beta)!(b-\beta)!
(c+\gamma)!(c-\gamma)! \right ] ^{1/2} \\
&\times \sum_\nu (-1)^\nu \big [ (a-\alpha-\nu)! (c-b+\alpha+\nu)! 
(b+\beta-\nu)! \\
&\qquad\qquad (c-a-\beta+\nu)! \nu! (a+b-c-\nu)! \big ]^{-1}
\end{split}
\end{equation}
where the sum is taken over values of $\nu$ which lead to non-negative 
factorials and
\begin{equation}
\Delta(a,b,c)=\left [ { {(a+b-c)!(b+c-a)!(c+a-b)!}\over{(a+b+c+1)!} } 
\right ]^{1/2}
\end{equation}

\subsection{Implementation of BRDFs}

Including the source term of the surface the condition to be applied is
\begin{equation}
I(\bn) = \int_{\Omega_-} \gamma_r(\bn, \bnp) (I(\bnp) +I_\odot \delta 
(\bnp-{\bn}_\odot) -B_*) 
(\bnp . {-\mathbf{e}}_z) \, d \omega_{\bnp} +B_*
\end{equation}
for $\bn\in\Omega_+$. Here, $B_*$ is the isotropic Planckian radiance that is
emitted by a blackbody at the surface temperature. The form of the surface
emission term is a direct consequence of Kirchoff's law. 

Expanding this equation in spherical harmonics,
\begin{equation}
\begin{split}
\sum_{lm} I_{lm} Y_l^m (\bn) &=\int_{\Omega_-} \sum_{lm} \sum_{l'm'} 
\Gamma_{lml'm'} Y_l^m (\bn) Y_{l'}^{m'*} (\bnp) (\bnp . {-\mathbf{e}}_z) \\
& \left [ I_\odot \delta (\bnp-{\bn}_\odot) +\sum_{\lambda\mu} 
I_{\lambda\mu} Y_\lambda^\mu (\bnp) -B_*
\right ] \, d \omega_{\bnp} + B_*
\end{split}
\end{equation}
As this covers only the upper hemisphere Marshak's procedure should be
applied, so inner products with $\int_{\Omega_+} Y_L^{M*}
\ldots d\omega_{\bn}$ are formed. It is easiest to consider each term 
separately, so noting the symmetries of $\kappa_{ll'm}$ as
defined above,
\begin{equation}
\int_{\Omega_+} Y_L^{M*} (\bn) \sum_{lm} I_{lm} Y_l^m (\bn) \, d\omega_\bn
= \sum_l I_{lM} (-1)^{L+l} \kappa_{LlM}
\end{equation}
For the solar term,
\begin{equation}
\begin{split}
\int_{\Omega_+} Y_L^{M*} (\bn) &\int_{\Omega_-} \sum_{lm} \sum_{l'm'}
\Gamma_{lml'm'} Y_l^m (\bn) Y_{l'}^{m'*} (\bnp) (-\bnp.\mathbf{e}_z) \,
I_{\odot} \delta (\bnp - \bn_{\odot} ) \, d\omega_{\bnp} \, d\omega_{\bn} \\
&= I_{\odot} (-\bn_\odot.\mathbf{e}_z) \sum_{lm} \sum_{l'm'} \Gamma_{lml'm'}
Y_{l'}^{m'*} (\bn_\odot) \int_{\Omega_+} Y_L^{M*} (\bn) Y_l^m (\bn)
\, d\omega_{\bn} \\
&= I_{\odot} \mu_\odot \sum_l \sum_{l'} Y_{l'}^{m'*} (\bn_\odot) \Psi_{ll'M}
(-1)^{L+l} \kappa_{LlM}.
\end{split}
\end{equation}
For reflected diffuse radiation
{\small
\begin{equation}
\begin{split}
\int_{\Omega_+} & Y_L^{M*} (\bn) \int_{\Omega_-} \sum_{lm} \sum_{l'm'}
\Gamma_{lml'm'} Y_l^m (\bn) Y_{l'}^{m'*} (\bnp) (-\bnp.\mathbf{e}_z)
\sum_{\lambda\mu} I_{\lambda\mu} Y_{\lambda}^{\mu} (\bnp) 
\, d\omega_{\bnp} \, d\omega_{\bn} \\
&= \sum_{lm} \sum_{l'm'} \sum_{\lambda\mu} \Gamma_{lml'm'} I_{\lambda\mu}
. \int_{\Omega_+} Y_L^{M*} (\bn) Y_l^m (\bn) \, d\omega_{\bn} \\
& . \int_{\Omega_-} (-\bnp.\mathbf{e}_z) Y_{l'}^{m'*} (\bnp) 
Y_{\lambda}^{\mu} (\bnp) \, d\omega_{\bnp} \\
&= \sum_l \sum_{l'} \sum_\lambda \Psi_{ll'M} I_{\lambda M} (-1)^{L+l} 
\kappa_{LlM} \\
& . \int_{\Omega_-} Y_{\lambda}^{M} (\bnp) (-1) \left [
c_{l'M}^+ Y_{l'+1}^{M*} (\bnp) + c_{l'M}^- Y_{l'-1}^{M*} (\bnp) \right ]
\, d\omega_{\bnp} \\
&=\sum_\lambda I_{\lambda M} \sum_l \sum_{l'} \Psi_{ll'M} (-1)^{L+l+1}
\kappa_{LlM} \left [ c_{l'M}^+ \kappa_{l'+1, \lambda, M} + c_{l'M}^-
\kappa_{l'-1, \lambda, M} \right ] .
\end{split}
\end{equation}
}
For the Planckian term coupled to the BRDF,
{\small
\begin{equation}
\begin{split}
\int_{\Omega_+} Y_L^{M*} (\bn) &\int_{\Omega_-} \sum_{lm} \sum_{l'm'}
\Gamma_{lml'm'} Y_l^m (\bn) Y_{l'}^{m'*} (\bnp) (-\bnp.\mathbf{e}_z)
B_* \, d\omega_{\bnp} \, d\omega_{\bn} \\
&= B_* \sum_{lm} \sum_{l'm'} \Gamma_{lml'm'} . \int_{\Omega_+} Y_L^{M*} (\bn)
Y_l^m (\bn)\, d\omega_{\bn} . \\
\qquad \sqrt{4\pi\over 3} \int_{\Omega_-} - Y_1^0 (\bnp) 
Y_{l'}^{m'*} (\bnp) \, d\omega_{\bnp} \\
&= \sqrt{4\pi\over 3} B_* \sum_l \sum_{l'} 
\Psi_{ll'M} (-1)^{L+l+1} \kappa_{LlM} \kappa_{l'10}
\delta_{M0}.
\end{split}
\end{equation}
}
Finally, the black-body term gives
\begin{equation}
\int_{\Omega_+} Y_L^{M*} (\bn) B_* \, d\omega_{\bn} 
= \sqrt{4\pi} B_* \kappa_{L00} \delta_{0M} (-1)^L.
\end{equation}
Now note that $c_{l,m}^- =c_{l-1,m}^+$, that
$\Psi_{ll'm}= \sum_j \rho_j F_{jll'm}$, and that each term contains a 
factor of $(-1)^L$, which may be cancelled, so collecting terms, 
{\small
\begin{equation}
\begin{split}
\sum_{\lambda} & I_{\lambda M} \Biggl \{ (-1)^\lambda \kappa_{L\lambda M}
+ \sum_j \rho_j \sum_l \sum_{l'} (-1)^l F_{jll'M} \kappa_{LlM} \\
&. \left [ c_{l',M}^+ \kappa_{l'+1,\lambda,M} 
+ c_{l'-1,M}^+ \kappa_{l'-1,\lambda,M} \right ] \Biggr \} \\
&= I_\odot \mu_\odot \sum_j \rho_j \sum_{l'} Y_L^{M*} (\bn_\odot)
\left [ \sum_l (-1)^l \kappa_{LlM} F_{jll'M} \right ] \\
&= B_* \delta_{0M} \left [ \sqrt{4\pi} \kappa_{L00} 
+ \sqrt{4\pi\over 3} \sum_j \rho_j \sum_{l'}
\kappa_{l'10} \sum_l (-1)^l \kappa_{LlM} F_{jll'M} \right ].
\end{split}
\end{equation}
}
We define,
\begin{align}
\Xi_{jLl'M} &= \sum_l (-1)^l \kappa_{LlM} F_{jll'M} \\
\Phi_{jL\lambda M} &= \sum_{l'} \Xi_{jLl'M} \left [ 
c_{l',M}^+ \kappa_{l'+1,\lambda,M} +c_{l'-1,M}^+ \kappa_{l'-1,\lambda,M}
\right ] \\
\noalign{and} \\
\Lambda_{jL} &= \sqrt{4\pi\over 3} \sum_{l'} \kappa_{l'10} \Xi_{jLl'0}.
\end{align}
Hence,
\begin{equation}
\begin{split}
\sum_l & I_{lM} \left \{ (-1)^l \kappa_{LlM} + \sum_j \rho_j 
\Phi_{jLlM} \right \} 
= I_\odot \mu_\odot \sum_j \rho_j \sum_{l'} Y_{l'}^{M*} ( \bn_\odot)
\Xi_{jLl'M} \\
&+ B_* \delta_{0M} \left \{ \sqrt{4\pi} \kappa_{L00} 
+ \sum_j \rho_j \Lambda_{jL}
\right \}
\end{split}
\end{equation}
When multiple calculations are performed within the same band (so that
the constants $\rho_j$ remain fixed) it is useful to simplify a little
further by writing the equation as
\begin{equation}
\sum_l I_{lM} M^{(1)}_{LlM} =I_\odot M^{(2)}_{LM} + B_* \delta_{0M} M^{(3)}_L
\end{equation}
where $M^{(1,2,3)}$ are defined by the obvious identifications.


Now, the components of the spherical harmonics at the bottom of the
lowest layer will be given by
\begin{equation}
\begin{split}
I_{Nlm}(\tau_N)= \sum_k \left \{ u_{mNk}^- (-1)^{l+m} V_{lmNk} 
\vartheta_{Nk} + u_{mNk}^+ V_{lmNk} \right \}
+\check G_{lmN} 
\end{split}
\end{equation}
Substituting into the boundary condition and replacing $M$ by $m$ and $L$ by
$l'$ to match the normal notation used at the top boundary,
{\small
\begin{equation}
\begin{split}
\sum_k & u_{mNk}^- \left [ \vartheta_{Nk} \sum_l M_{l'lm}^{(1)} 
(-1)^{(l+m)} V_{lmNk} 
\right ] + u_{mNk}^+ \left [ \sum_l M_{l'lm}^{(1)} 
V_{lmNk} \right ] \\
&= I_\odot M_{l'm}^{(2)} + B_* \delta_{0m} M_{l'}^{(3)} 
- \sum_l M_{l'lm}^{(1)} \check G_{lmN}
\end{split}
\end{equation}
}


\section{Numerical Implementation}

In principle, the coefficients $I_{lm}$ must be calculated for the range
$0\le l \le L$ and $-l \le m \le l$; moreover, these coefficients are complex.
In practice, the storage required can be reduced by making use of the
various symmetries of the coefficients.

First note that $I\in\mathbb{R}$, so that $I=I^*$ and
\begin{equation}
\sum I_{lm} Y_l^m = \sum I_{lm}^* Y_l^{m*} = \sum I_{lm}^* (-1)^m Y_l^{-m}
= \sum I_{l,-m}^* (-1)^{-m} Y_l^{m}
\end{equation}
whence
\begin{equation}
I_{l,-m}= (-1)^m I_{lm}^* ,
\end{equation}
so coefficients with $m<0$ may be found by symmetry.

The complex nature of $Y_l^m$ appears only through the factor $e^{im\phi}$, so
we may define $Y_l^m=\Upsilon_l^m e^{im\phi}$ where $\Upsilon_l^m \in 
\mathbb{R}$. With the restrictions imposed on the BRDF above which forbid
the coupling of harmonics with different azimuthal orders, the complex
nature of $I_{lm}$ appears only through a factor $e^{-im\phi_\odot}$, so
we write $I_{lm}=C_{lm}e^{-im\phi_\odot}$ (in the case of IR radiation
where the radiance is azimuthally symmetric the value of $\phi_\odot$ is 
immaterial). Hence, if $m>0$
\begin{equation}
\begin{split}
I_{lm} Y_l^m + I_{l,-m} Y_l^{-m} &= C_{lm} e^{-im\phi_\odot} 
\Upsilon_l^m e^{im\phi} \\
&+ (-1)^m C_{lm} e^{im\phi_\odot}
\Upsilon_l^m (-1)^m e^{-im\phi} \\
&= 2C_{lm} \Upsilon_l^m \cos m(\phi-\phi_\odot)
\end{split}
\end{equation}


\section{Increasing the Speed of Computation}

So far, we have described the method of calculating the amplitudes of the
spherical harmonics. This is perfectly adequate to calculate fluxes, but
it converges very slowly when calculating radiances. The source function
technique, due originally to Kourganoff (1955), can be used to circumvent
this problem. In this technique, the radiance is calculated by integrating
along a ray, using the direct solution by spherical harmonics to represent
the scattered radiation (the term in the equation of transfer involving
an integral over the phase function): this can be looked upon as a kind
of iterated solution of the problem. This technique has many points in
common with a technique for reducing the number of harmonics required
to obtain a given accuracy, as described next.

When the equation of transfer is solved using spherical harmonics, a high
order of truncation may be required to represent the radiance field. These
higher orders are principally required to represent singly scattered
radiation; but singly scattered radiances can be calculated more simply
than the multiply scattered radiances, so it is sensible to examine ways
of separating the singly scattered component of the radiance so that a
rather lower order of truncation can be used to calculate the multiply 
scattered radiances. \cite{Nakajima88} have considered various approximations
of this form, particularly for the case of optically thin layers. They
eventually derived a method they refer to as IMS, but this 
is insufficiently general
for optically thick layers because it exhibits an instability, and is
therefore inappropriate for use in a code that will be used in a GCM.
We will therefore adopt the method which they refer to as TMS, which performs
almost as well as IMS except very close to the forward direction. They present 
only a very sketchy derivation of the method and no justification, 
so it is useful to derive it more fulsomely here.
The idea is to retain the full phase function in the calculation of single
scattering, but to use the rescaled truncated phase function for multiple
scattering. In this section we shall use the caret to denote rescaled 
quantities. Under rescaling the phase function is rewritten as
\begin{equation}
P(\bnp, \bn) = 4 \pi f \delta(\bnp-\bn) + (1-f) \hat P (\bnp, \bn)
\end{equation}
Splitting the diffuse and direct beams as usual we obtain the following
equation for the diffuse radiance, $I$,
\begin{equation}
\begin{split}
(\bn . \nabla) I(\bn) & = -(k^{(s)} + k^{(a)}) I (\bn) \\
& + { k^{(s)} \over
{4 \pi} } \int_{\Omega} I(\bnp) \left ( 4 \pi f \delta(\bnp-\bn) 
+ (1-f) \hat P(\bnp, \bn) \right ) \, d\omega_{\bnp} \\
& + { k^{(s)} \over
{4 \pi} } \int_{\Omega} I_\odot(\bnp) P(\bnp, \bn) \, d\omega_{\bnp}
\end{split}
\end{equation}
Notice here that $I_\odot$ refers to the true solar beam, calculated without
rescaling, since we have not used a rescaled phase function for single 
scattering. Writing this equation in terms of the rescaled optical propeties
we have
\begin{equation}
\mu {{dI}\over{d\hat\tau}} = I -{\hat\omega\over{4\pi}} \int_\Omega I \hat P
\, d\omega_\bnp - {\hat\omega\over{4\pi (1-f)}} \int_\Omega I_\odot P
\, d\omega_\bnp
\end{equation}
To obtain the exact approximation of \cite{Nakajima88} we replace$I$ in
the first integral with $\hat I_T$, the truncated diffuse radiance and
$\hat P$ with $\hat P_T$, the truncated rescaled phase function.

This sits very easily with the method of solution for radiances originally
suggested by \cite{Kourganov55}, in which we regard the spherical harmonic
solution as defining the source function for the diffuse radiation in the 
above equation, but it is actually more convenient to proceed very slightly
differently from \cite{Nakajima88}. To be precise, we first solve the 
truncated rescaled equation
using spherical harmonics to get a rescaled truncated diffuse radiance, $\hat 
I_T$, and a rescaled direct radiance, $\hat I_\odot$. Our first approximation
to the true (unrescaled) diffuse radiance is thus
\begin{equation}
\tilde I = \hat I_T + \hat I_\odot -I_\odot,
\end{equation}
allowing for the change in the definition of the direct beam when
switching from rescaled to unrescaled radiances. Now, it is necessary to
be very careful in the treatment of the direct terms: errors may arise
either in the form of $\delta$-functions in the solar direction, or as
diffused errors at other angles. These fast methods are not accurate
close to the solar direction, so in practice it turns out to be better
to drop the contribution to the diffuse radiance from the change in the
definition of the solar beam, which concentrates errors around the solar
peak, rather than concentrating some there and diffusing others, so we
take just $\hat I_T$ as the diffuse radiance.


Substituting this into the second term of the above equation we get
\begin{equation}
\mu {{dI}\over{d\hat\tau}} = I 
-{\hat\omega\over{4\pi}} \int_\Omega 
\hat I_T  \hat P \, d\omega_\bnp 
- {\hat\omega\over{4\pi}} \int_\Omega I_\odot {P\over{(1-f)}} 
\, d\omega_\bnp
\end{equation}
$\hat I_T$ involves only harmonics up to the order of $\hat P_T$, so we
may use the truncated phase function when multiplying it, hence
\begin{equation}
\mu {{dI}\over{d\hat\tau}} = I
-{\hat\omega\over{4\pi}} \int_\Omega
\hat I_T \hat P_T \, d\omega_\bnp
-{\hat\omega\over{4\pi (1-f)}} \int_\Omega
I_\odot P \, d\omega_\bnp
\end{equation}
From the algorithmic point of view, we initially solve the rescaled problem
and finally perform a separate calculation of the unrescaled solar
contribution.

To develop the mathematics for this carefully, we introduce the sets of
spherical orders $\cal F$ and $\cal T$, for the full set of spherical 
orders used in the final expression and for the truncated set used in 
the direct solution, which is written
\begin{equation}
\hat I_T(\bn, \tau) = \sum_{(l,m)\in {\cal T}} Q_{l,m} (\tau) Y_l^m (\bn)
\end{equation}
Moreover,
\begin{align}
 \hat P_T (\bnp, \bn) &= 4\pi \sum_{(l,m)\in {\cal T}} \hat g_l Y_l^{m*} (\bnp)
Y_l^m (\bn) \\
\noalign{and}
 \hat P (\bnp, \bn) &= 4\pi \sum_{(l,m)\in {\cal F}} \hat g_l Y_l^{m*} (\bnp)
Y_l^m (\bn) \\
\end{align}
Substituting these expressions into the equation of transfer we obtain
\begin{equation}
\begin{split}
\mu {{dI}\over{d\hat\tau}} = I
&-\hat\omega(\hat\tau) \sum_{\cal T} \hat g_l(\hat \tau) Q_{lm} (\hat \tau) 
Y_l^m(\bn) \\
&-\hat\omega(\hat\tau) \hat I_\odot (\hat \tau) 
\sum_{\cal F} \hat g_l(\hat \tau) 
Y_l^{m*} (\bn_\odot) Y_l^m(\bn)
\end{split}
\end{equation}
Integrating with respect to optical
depth between $\hat\Delta^-$ and $\hat\Delta^+$, we obtain
\begin{equation}
\begin{split}
I(\bn, \hat\Delta^+)  &= I(\bn, \hat\Delta^-) e^{(\hat\Delta^+/\mu-\hat\Delta^-/\mu)} \\
&  -{1\over\mu} e^{\hat\Delta^+/\mu}\sum_{(l,m)\in{\cal T}} Y_l^m (\bn) 
\int_{\hat\Delta^-}^{\hat\Delta^+}
\hat \omega(\hat \tau) \hat g_l(\hat\tau) Q_{l,m} (\hat\tau) 
e^{-\hat\tau/\mu} \, d\hat\tau \\
& -{1\over \mu} e^{\hat\Delta^+/\mu}\sum_{(l,m)\in{\cal F}} 
Y_l^{m*} (\bn_\odot) Y_l^m (\bn)
\int_{\hat\Delta^-}^{\hat\Delta^+} \hat \omega(\hat \tau) \hat g_l(\hat\tau)\hat 
I_\odot (\hat\tau) e^{-\hat\tau/\mu} \, d\hat\tau 
\end{split}
\end{equation}
In the code the integrals on the right will be evaluated separately for
each layer, in which the optical propeties will be taken as fixed, so if
${\cal I}$ denotes the set of those layers that contain regions of optical
depth between $\hat\Delta^-$ and $\hat\Delta^+$ we may write
\begin{equation}
\begin{split}
I(\bn, \hat\Delta^+)  &= I(\bn, \hat\Delta^-) e^{(\hat\Delta^+/\mu-\hat\Delta^-/\mu)} \\
&  +\sum_{i\in\cal I} \sum_{(l,m)\in{\cal T}} 
\hat\omega_i \hat g_{li} Y_l^m (\bn) A_{ilm}
+\sum_{i\in\cal I} \sum_{l\in{{\cal F}_L}} { {(2l+1)}\over{4\pi}}
\hat\omega_i \hat g_{li} P_l (\bn_\odot . \bn) B_i
\end{split}
\end{equation} 
where $A_{ilm}$ and $B_i$ denote the contributions from the individual
layers with the obvious identifiation. In the case of $B_i$ we have used
standard results to reexpress the spherical harmonics as Legendre 
polynomials. Each of these contributions is evaluated separately, 
setting the limits of integration to $\hat\Delta_i^-$ and $\hat\Delta_i^+$ 
which will normally mark the edges of the layer, though not in the case of 
the layer containing the level where we seek the radiance may be within it. 
Now,
\begin{equation}
\begin{split}
Q_{ilm} (\hat\tau) &= Z_{ilm} e^{-(\hat\tau -\hat\Delta_{i-1})/\mu_0} \\
&+\sum_k \biggl [ 
\begin{aligned}[t]
& u_{mik}^+ V_{ikl} e^{-(\hat\Delta_i-\hat\tau)/\mu_{mik}} \\
+& u_{mik}^- V_{ikl} (-1)^{(l+m)} e^{-(\hat\tau-\hat\Delta_{i-1})/\mu_{mik}} 
\biggr ]
\end{aligned}
\end{split}
\end{equation}
so the contribution to the first integral from the $i$th layer is
\begin{equation}
\begin{split}
A_{ilm} &= -{1\over\mu}Z_{ilm} e^{(\hat\Delta_{i-1}/\mu_0+\hat\Delta^+/\mu)}
\int_{\hat\Delta_i^-}^{\hat\Delta_i^+} 
e^{-\hat\tau (1/\mu+1/\mu_0)} \, d\hat\tau \\
&-{1\over\mu}\sum_k  u_{mik}^+ V_{ikl} 
e^{(\hat\Delta^+/\mu-\hat\Delta_i/\mu_{mik})} 
\int_{\hat\Delta_i^-}^{\hat\Delta_i^+} 
e^{\hat\tau (-1/\mu+1/\mu_{mik})}\, d\hat\tau
\\
&- {1\over\mu}\sum_k u_{mik}^- V_{ikl} (-1)^{(l+m)} 
e^{-(\hat\Delta^+/\mu+\hat\Delta_{i-1}/\mu_{mik})}
\int_{\hat\Delta_i^-}^{\hat\Delta_i^+} 
e^{-\hat\tau (1/\mu+1/\mu_{mik})}\, d\hat\tau
\\
&= Z_{ilm} {\mu_0\over{\mu+\mu_0}}
\begin{aligned}[t] 
\Biggl \{ &\exp \left ( {{\hat\Delta^+-\hat\Delta_i^+}\over\mu}
+{{\hat\Delta_{i-1}-\hat\Delta_i^+}\over\mu_0} \right) \\
- &\exp \left ( {{\hat\Delta^+-\hat\Delta_i^-}\over\mu}
+{{\hat\Delta_{i-1}-\hat\Delta_i^-}\over\mu_0} \right) \Biggr \} 
\end{aligned} \\
&+ \sum_k u_{ikm}^+ V_{ikl} {\mu_{mik}\over{\mu_{mik}-\mu}}
\begin{aligned}[t] 
\Biggl \{ &\exp \left ( {{\hat\Delta^+-\hat\Delta_i^+}\over\mu}
+{{\hat\Delta_i^+-\hat\Delta_i}\over\mu_{mik}} \right) \\
- &\exp \left ( {{\hat\Delta^+-\hat\Delta_i^-}\over\mu}
+{{\hat\Delta_i^--\hat\Delta_i}\over\mu_{mik}} \right) \Biggr \} 
\end{aligned} \\
&+ \sum_k u_{ikm}^- (-1)^{(l+m)} V_{ikl} 
{\mu_{mik}\over{\mu_{mik}+\mu}}
\begin{aligned}[t] 
\Biggl \{ &\exp \left ( {{\hat\Delta^+-\hat\Delta_i^+}\over\mu}
+{{\hat\Delta_{i-1}-\hat\Delta_i^+}\over\mu_{mik}} \right) \\
- &\exp \left ( {{\hat\Delta^+-\hat\Delta_i^-}\over\mu}
+{{\hat\Delta_{i-1}-\hat\Delta_i^-}\over\mu_{mik}} \right) \Biggr \} .
\end{aligned}
\end{split}
\end{equation}
Likewise, from the second term we obtain a contribution
\begin{equation}
\begin{split}
B_i &= -{1\over\mu} \hat I_{\odot i-1}
e^{\left(\hat\Delta_{i-1}/\mu_0+\hat\Delta^+/\mu\right)}
\int_{\hat\Delta_i^-}^{\hat\Delta_i^+} e^{-\hat\tau (1/\mu+1/\mu_0)} 
\, d\hat\tau \\
&= \hat I_{\odot i-1}
{\mu_0\over{\mu+\mu_0}} 
\begin{aligned}[t]
\Biggl \{ &\exp \left ( {{\hat\Delta^+-\hat\Delta_i^+}\over\mu}
+{{\hat\Delta_{i-1}-\hat\Delta_i^+}\over\mu_0} \right) \\
-& \exp \left ( {{\hat\Delta^+-\hat\Delta_i^-}\over\mu}
+{{\hat\Delta_{i-1}-\hat\Delta_i^-}\over\mu_0} \right) \Biggr \} 
\end{aligned}
\end{split}
\end{equation}

There is a problem with ill-conditioning when $\mu\rightarrow\mu_0$ or
$\mu\rightarrow\pm\mu_{mik}$. Each geometrical factor which may produce
ill-conditioning is of the form
\begin{equation}
\begin{split}
G&={{\tilde\mu}\over{\tilde\mu-\mu}} \left \{ e^{(-s_n+\hat\tau_i/{\tilde\mu}}
-e^{-s_f} \right \} \\
&= \hat\mu e^{-s_n} { {e^{\hat\tau_i/{\tilde\mu}} -e^{\hat\tau_i/{\mu}}}
\over{\tilde\mu-\mu}}
\end{split}
\end{equation}
where $\tilde\mu$ stands generically for $\mu_0$ or $\pm\mu_{mik}$ and $s_n$ and
$s_f$ represent the slant depths from the observing level to the nearer and
farther boundaries of the layer. As $\mu\rightarrow\tilde\mu$,
\begin{equation}
\begin{split}
G&\rightarrow {{\tilde\mu}\over{\tilde\mu-\mu}} e^{-s_n+\hat\tau/{\tilde\mu}} 
\left ( 1 - e^{-\hat\tau(1/\mu-1/\tilde\mu)} \right ) \\
&\rightarrow {\hat\tau\over\mu} e^{-s_n+\hat\tau/{\tilde\mu}}
\end{split}
\end{equation}
Now reacall L'H\^opital's rule that if $\lim_{x\rightarrow 0} f(x), g(x)=0$
then 
\begin{equation}
\lim_{x\rightarrow 0} {{f(x)}\over{g(x)}}
=\lim_{x\rightarrow 0} {{f'(x)}\over{g'(x)}}.
\end{equation}
Consequently, 
\begin{equation}
\lim_{x\rightarrow 0} {{(f(x)+\eta(x)f'(x))}\over{(g(x)+\eta(x)g'(x))}}
=\lim_{x\rightarrow 0} {{f(x)}\over{g(x)}}. 
\end{equation}
Supposing that $g'(0)\ne 0$, it follows 
that if we arrange that $\eta(x)$ is small compared to $g(x)$ except in
the neighbourhood of $x=0$, we have an expression for the quotient which
does not become indeterminate as $x\rightarrow 0$ and will be approximately
accurate for
all values of $x$. One possible choice for $\eta$ is $\eta(x)=\epsilon/
(x+\sqrt\epsilon)$ where $\epsilon$ is the smallest number such that
$1-\epsilon\ne1$ to the computer's precision. This will introduce errors of
$O(\sqrt\epsilon)$ when $x=O(\sqrt\epsilon)$. In the present case we define
\begin{equation}
\eta={\epsilon\over{(\tilde\mu-\mu)+\mbox{sgn}(\tilde\mu-\mu)\sqrt\epsilon}}
\end{equation}
and put
\begin{equation}
G\approx \tilde\mu { { \left ( 1 - {{\eta\tau}\over{\mu\tilde\mu}} \right )
e^{-(s_n+\hat\tau/\tilde\mu)}-e^{-s_f} } \over {\tilde\mu-\mu+\eta} }
\end{equation}



